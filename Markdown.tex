% !TEX TS-program = LaTeX+Markdown
\documentclass{article}

% Settings specific to this document.
\usepackage[english]{babel}
\usepackage[T1]{fontenc}
\usepackage[utf8]{inputenc}
\usepackage[osf,sc]{mathpazo}
\usepackage[scaled=0.85]{beramono}
\usepackage[parfill]{parskip}
\linespread{1.05}

% Pandoc assumes these packages are present. However, many can be excluded
% if your document doesn't use all features of Pandoc Markdown.
\usepackage{amsmath,amssymb}
\usepackage{booktabs}
\usepackage{fancyvrb}
\usepackage{graphicx}
\usepackage[colorlinks]{hyperref}
\usepackage{ifluatex,ifxetex}
\usepackage{listings}
\usepackage{longtable}
\usepackage[normalem]{ulem}
\usepackage{url}
\ifxetex
  \usepackage{fontspec}
  \usepackage{xltxtra}
  \usepackage{xunicode}
\fi
\ifluatex
  \usepackage{fontspec}
\fi

% Include markdown.sty.
\usepackage{markdown}

% Set up formatting of Haskell listings.
\lstset{
  language=Haskell,
  basicstyle=\ttfamily,
  xleftmargin=2ex,
  keywordstyle=,
  flexiblecolumns=false,
  literate={->}{{$\rightarrow$}}2 {=>}{{$\Rightarrow$}}2 {\\}{{$\lambda$}}1
}

\begin{document}

\title{Pandoc-rendered Markdown in \LaTeX}
\author{Eric Walkingshaw \\ \small eric@walkingshaw.net}
\date{}
\maketitle

\begin{markdown}
The `markdown.sty` package allows you to write [Markdown]-formatted text inside
of \LaTeX\ using the `markdown` environment and the `\markd` command. In fact,
the text you're reading *right now* is [Pandoc]-rendered Markdown.^[That is,
unless you're reading the source code, in which case it's not yet rendered. :-)] 

[Markdown]: http://daringfireball.net/projects/markdown/
[Pandoc]: http://johnmacfarlane.net/pandoc/
\end{markdown}

The \texttt{markdown} environment is useful for blocks of Markdown-formatted
text, such as the above paragraph. These blocks must be reasonably
self-contained. For example, reference-style links (used above) must be defined
within the same environment they are used.

Using the \markd{`\markd`} command, you can include short snippets of Markdown
\markd{*within*} regular text. This paragraph contains several examples. A use
of the \markd{`\markd`} command cannot be split across multiple
lines.\markd{^[Due to a limitation of the `xparse` package.]}

\begin{markdown}
Within the `markdown` environment, you can use all of the usual Markdown
features plus Pandoc's [neat extensions][Exts]. We've seen footnotes, links,
and basic formatting already. Other useful features include in-line `code`
`spans` and pre-formatted code blocks.

      main() {
        printf("Hello, world!");
      }
 
Also, itemized lists:
 
 * Here's an item with a plain bullet.
     i. And a nested item with a Roman numeral bullet.
          a) And a nested--nested item with a different kind of bullet.
 
An especially relevant Pandoc extension is support for embedded
\LaTeX\ macros, including references (see Table \ref{tbl:dep}) and
`$`-delimited math: $e^{i\pi} = -1$.

[Exts]: http://johnmacfarlane.net/pandoc/README.html#pandocs-markdown
\end{markdown}
%stopzone % just to fix vim's syntax highlighting ...

\begin{markdown}

For some Markdown features, the \LaTeX\ code that Pandoc generates relies on
external packages. These can be included in the usual way with `\usepackage`.
The Pandoc documentation contains a [complete list][Pkgs] of the packages it
assumes are loaded, or you can also find them at the top of the source that
generated this document. If your document doesn't use every feature of Pandoc,
you can omit some of these packages. A few of the dependencies I know about are
listed in Table \ref{tbl:dep}. There are many others! Mostly this was just an
excuse to show off tables and references in Markdown.

[Pkgs]: http://johnmacfarlane.net/pandoc/README.html#creating-a-pdf
\end{markdown}

\begin{figure}[t]
\begin{markdown}
**Feature**          **Dependencies**
-------------        -----------------------
Images               `graphicx`
Links                `hyperref`, `url`
Strikethrough        `ulem`
Syntax highlighting  `listings` (requires `--listings`)
Tables               `booktabs`, `longtable`

Table: \LaTeX\ package dependencies of Markdown features. \label{tbl:dep} 
\end{markdown}
\end{figure}

\begin{markdown}
Both the `markdown` environment and the `\markd` command accept an optional
argument containing additional arguments to pass to Pandoc. One use of this is
to add syntax highlighting and character replacement to code blocks with
Pandoc's `--listings` option, as illustrated in the following Haskell code.
\end{markdown}

\begin{markdown}[--listings]
~~~haskell
plusOne :: Num a => [a] -> [a]
plusOne = map (\a -> a + 1)
~~~
\end{markdown}

% TODO Add description of TeXShop engine.
% TODO Conclude gracefully.

\section*{Acknowledgments}
\label{sec:ack}

\begin{markdown}
Thank you to the StackExchange user [G. Poore][profile] who solved all of the
hard bits in his excellent answers to other users' questions. After having the
idea, I immediately found his [concise solution][sol], which is the basis for
this package. Later, [another answer][opt] of his helped me figure out how to
add optional arguments to a verbatim environment.

[profile]: http://tex.stackexchange.com/users/10742/g-poore
[sol]: http://tex.stackexchange.com/a/101731
[opt]: http://tex.stackexchange.com/a/111990
\end{markdown}

\end{document}
